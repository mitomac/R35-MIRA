
\section*{Abstract}
Our research is focused on understanding how the local chromatin environment regulates DNA-templated processes like transcription and DNA replication.  We have developed and pioneered the use of factor-agnostic approaches to map chromatin occupancy at nucleotide resolution across the \scer\ genome.  We have used this approach to provide mechanistic insights into how the local chromatin environment mediates origin selection and activation. In order to identify the chromatin changes that occur with helicase activation at the onset of S-phase, we depleted cells of functional polymerase alpha to prevent the priming of DNA synthesis.  Activation of the helicase in the absence of priming not only resulted in the local disruption of chromatin, but also resulted in the uncoupling of the helicase from DNA synthesis and the unwinding of approximately 1 kb of DNA surrounding each activated origin.   We will identify the mechanism(s) which regulate helicase progression in the absence of DNA replication with a focus on sequence, topological and extrinsic signaling.  Understanding the 'dead man switch'....   A consequence of the helicase traveling away from the origin in each direction and stalling is that upon restoration of pol alpha priming the helicase is oriented to travel away from the origin and thus will leave an unreplicated gap.  These unreplicated gaps are a unique molecular intermediate that are analogous to intermediates predicted to occur from termination defects and provide a unique opportunity to identify factors involved in their resolution.  genetic and epigenetic.  We will also examine the role of specific histone chaperones in the spatiotemporal deposition of nucleosomes behind the DNA replication fork and how specific chromatin features like transcription may impact the maturation process in a locus specific manner.   



in G1 and subsequent origin activation as cell enter S-phase.  We w

% Our research is focused on elucidating the mechanisms by which the local chromatin environment influences DNA-templated processes including DNA replication, transcription and DNA repair.  While considerable progress has been made in our understanding of the mechanisms that direct DNA replication \invitro, we know very little about how start sites of DNA replication (origins) are selected and regulated in the context of the chromosome.  The genomic approaches that my research group have pioneered have provided new insights into the mechanisms by which the local chromatin state and structure (nucleosome and transcription factor occupancy) influences key steps in regulating the DNA replication program in multiple species including \scer and \dros.  We have recently developed a novel approach to `footprint' a eukaryotic genome -- simultaneously revealing genome-wide occupancy of DNA for both nucleosomes and smaller DNA binding factors (\eg initiation and transcription factors).  Unlike biochemical reconstitution experiments utilizing one or two defined DNA templates, we are able to comprehensively view the cell cycle regulated cascade of chromatin changes that occur surrounding each origin of replication in the yeast genome.  Our future research will focus on identifying and characterizing the chromatin mediated events required for initiation of DNA replication following helicase loading.  We will also investigate how chromatin structure is re-established throughout the genome to preserve epigenetic integrity following passage of the DNA replication fork.  DNA replication is also a potent source of double-stranded breaks (DSB) which, if not repaired, may lead to genomic instability.  We are uniquely positioned to identify and understand the dynamics of chromatin structure following the induction of site-specific DSBs and their subsequent repair by homologous recombination or non-homologous end joining. Finally, in collaboration with the Hartemink laboratory (Duke, CS) we are using synchronous populations of yeast proceeding through the cell cycle to develop robust statistical approaches that will enable us to model cell cycle-dependent changes in gene expression from chromatin occupancy data. 
