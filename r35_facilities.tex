\section*{Facilities}
\bheading{Laboratory}
Dr.  MacAlpine´s laboratory occupies around 1,200 square feet in the Department of Pharmacology and Cancer Biology (PCB) within the Levine Science Research Center (LSRC).  The lab space has benches for 12 investigators, a fume hood, and animal procedures.  Each bench has network ports available for computers connected to the Internet.  Additional space is available for large or shared equipment and tissue culture.

The environment at Duke is uniquely suited for these studies.  As a member of the Duke Cancer Institute  (DCI) and Center for Genome and Computational Biology (GCB), Dr. MacAlpine has access to a variety of shared resources including multiple next-generation sequencing platforms (Illumina, PacBio and Nanopore) and cluster computing environments.  The campus boasts a strong and interactive cadre of faculty studying various aspects of transcriptional regulation and nucleic acid metabolism, as well as a diversity of relevant graduate programs. MacAlpine's research group is composed of students from the University Program in Genetics and Genomics, Computational Biology and Bioinformatics, Pharmacology, and Molecular Cancer Biology.

\bheading{Clinical}
Not applicable.

\bheading{Animal}
Not applicable.

\bheading{Computer}
The MacAlpine computer facilities are located adjacent to the MacAlpine research laboratory.  These currently include an 16 processor Xenon Server with 32GB of RAM and 8TB of local storage.  Dedicated Linux and Mac workstations are available for data analysis.   Laser printers and scanners are also available. All computational workstations, servers, and disk space owned by GCB Investigators are managed by the GCB information technology (IT) group directed by Dr. Hilmar Lapp and staffed with four application and database developers and system administrators.  The GCB provides shared resources consisting of several compute servers connected to a SAN storage system with a total capacity of about 800TB of disk storage that provides fast access to data, with built-in redundancy and daily backups to protect against data loss.  Individual lab servers and disk space are integrated into the GCB infrastructure and are completely managed by the IT group.  
 
\bheading{Office}
 Dr. MacAlpine´s office is located in the PCB Department, also within the LSRC building.  His office space is immediately adjacent to his laboratory.
 
 
\bheading{Other}
 Duke Genome Sequencing Shared Resource  The Genome Sequencing Shared Resource (GSSR)  has operated and made available to Duke researchers a variety of genome-sequencing instruments and related support equipment and analysis for more than 10 years.  Its faculty director is Dr. Greg Wray, and it is staffed full-time by its associate director, Dr. Olivier Fedrigo, and six other employees: a bioinformatics specialist, an operations administrator, and four lab technicians.  The GSSR has over 2,000 square feet of lab space, equipment rooms, and office space. All instruments are housed and maintained in the facility and are available on a fee-for-service and cost-recovery basis for academic researchers.

Dr. MacAlpine has been a client for several years, including for his production role in the modENCODE project, and has excellent working relationships with GSSR staff. The GSSR has a wide range of sequencing platforms available including Illumina MiSeq and HiSeq 2000 and 2500 sequencers, IonTorrent PGM and Proton sequencers, and a PacBio RS~II sequencer.  The Illumina instruments will provide the majority of our sequencing reads, but diverse other platforms are available for any sort of validation we might require along the way.  Specifically, the GSSR provides a wide range of next-generation library preparations and sequencing services including DNA-seq, RNA-seq, ChIP-seq, MNase-seq, DNase-seq, smRNA-seq, and ATAC-seq, as well as mate-pair and targeted (exome, amplicon, gene panels) sequencing.  Ancillary equipment includes a Beckman Coulter Biomek~FX liquid handling robot, a Beckman Coulter Z2 particle counter, a Genomic Solutions HydroShear fragmentation apparatus, Agilent 2100 Bioanalyzer and 2200 TapeStation QC instruments, an Apollo 324 NGS library prep system, a Qiagen TissueLyser, a Covaris E-Series focused-ultrasonicator, a Life Technologies StepOnePlus RT-PCR instrument, and two Qubit 2.0 fluorometers for nucleic acid quantitation, along with three servers for next-generation sequencing data primary analyses and two additional compute servers, one with 24 CPU-cores and 256GB RAM for secondary and tertiary bioinformatics analysis.

Duke Scalable Computing Support Center (SCSC)  To support researchers with high-performance computing needs, Duke established the Scalable Computing Support Center (SCSC).  The center is staffed with a variety of computational experts, providing researchers with a broad range of support from algorithm development, performance improvement, code parallelization, and data visualization.  In conjunction with the Office of Information Technology (OIT), SCSC maintains a Linux cluster called the Duke Shared Cluster Resource (DSCR).  The power, cooling, maintenance, and systems administration for all of the equipment in the DSCR, including researcher-owned machines, is handled by OIT staff.  At present, the DSCR consists of around 4,300 CPU-cores, and these--plus 720 additional CS cores and 640 additional GCB cores--are available to the faculty on this project.
 
Duke Data Commons (NIH 1S10OD018164-01) Duke Research Computing conceived and has managed a large 1.5 petabyte storage installation of EMC Isilon equipment to support data-producing core facilities in the life sciences. The project, funded by the National Institutes of Health in fall 2014 (NIH 1S10OD018164-01), began operation in November 2014. This initiative was designed not merely to provide a place to put data, but is the beginning of a data commons for life sciences research, which has experienced explosive growth in data that are useful for researchers. Thus, as the project matures, the data storage equipment will serve as a foundation for building solutions to data management challenges, especially those arising from research using large, complex, multi-dimensional data. Duke researchers and their collaborators can use the data commons by using one or more of the core facilities to extract data for their projects. At present, these data are mostly molecular profiles (DNA or protein sequence and RNA analysis), though the data commons is also used for light microscopy and data under analysis by Duke's ¨`Omics Analysis Core.¨ Researchers in the life sciences with extraordinary data storage requirements also use the resource, though the resource is particularly targeted to meet needs of NIH-sponsored researchers, as of course are all NIH “Shared Instrumentation Grants” (S10). Duke Research Computing and Duke's Office of Information Technology have created a plan for sustaining storage capacity and for developing computational resources that can be attached to the storage. For example, data storage for particularly sensitive data – such as patient data regulated by HIPAA/HITECH – has been isolated and attached to Duke's Protected Research Network where computational analysis of the data can be conducted safely and securely. A project underway in Duke's School of Medicine will begin to use the storage for data requiring “high provenance” – particularly for data collected in biomedical research.
 




